\documentclass[a4paper,oneside,12pt,titlepage]{scrartcl}   %Grundeinstellungen
%
% Technische Universitaet Berlin
% Fachgebiet fuer Elektronische Mess- und Diagnosetechnik
% Vorlage fuer Protokoll zum Praktikum Grundlagen der elektronischen Messtechnik
% Berlin Oktober 2012
% Erstellt von Elvira Fleig und Sebastian Nowoisky
%
%Pakete bei Bedarf auskommentieren
\usepackage[ngerman]{babel}
\usepackage[utf8]{inputenc}
%!!! Es könnte sein, dass ihr eine andere Kodierung braucht. Wenn ihr Probleme mit der Anzeige von Umlauten habt, probiert es mit "latin1" (Windows), "applemac" (Mac OS) oder "utf8" (Unix).!!!
\usepackage{amsfonts}
\usepackage{amsmath}
\usepackage{amssymb}
%\usepackage{siunitx}
%\usepackage{SIunits}
\usepackage[amssymb]{SIunits} 

\renewcommand{\familydefault}{\sfdefault} 
\usepackage{graphicx} 
\usepackage{subfigure}                         
\usepackage{float}
\usepackage[section]{placeins}
\renewcommand{\topfraction}{0.85}
\renewcommand{\textfraction}{0.1}
\renewcommand{\floatpagefraction}{0.75}
%BibTex nützlich zum Zitieren. Kann bei Nichtgebrauch auch auskommentiert werden.
\usepackage{cite}
%Colorpackage
\usepackage[usenames,dvipsnames]{xcolor}
%Listings um Scilabquellcode leichter einzufügen

\usepackage{listings}

\lstset{%
	basicstyle=\scriptsize\ttfamily,
	keywordstyle=\bfseries\ttfamily\color{NavyBlue},
	stringstyle=\color{Rhodamine}\ttfamily,
	commentstyle=\color{Green}\ttfamily,
	emph={square}, 
	emphstyle=\color{blue}\texttt,
	emph={[2]root,base},
	emphstyle={[2]\color{yac}\texttt},
	language=Python,%
	tabsize=2,%
	basicstyle=\footnotesize\ttfamily,%
	numbers=left,%
	numberfirstline,%
	breaklines=true,%
	breakatwhitespace=true,%
	linewidth=\textwidth,%
	xleftmargin=0.075\textwidth,%
	frame=tlrb,%
	captionpos=b%
}


%Header Definitionen
\usepackage{fancyhdr}
\renewcommand{\headrulewidth}{0.5pt}
\renewcommand{\footrulewidth}{0.5pt}
\voffset26pt 
\parskip6pt
\usepackage{setspace}
\onehalfspacing

%___ÜBERSICHT_____________________________________________________

%Hier beginnt das Dokument
\begin{document}
	\pagenumbering{roman}
	
	\titlehead
	{
		\begin{tabular}{ll}
			\begin{minipage}{0.5\textwidth}
				\begin{figure}[H]
					\raggedright
					
				\end{figure}
			\end{minipage}
			\begin{minipage}{0.5\textwidth}
				\begin{figure}[H]
					\raggedleft
					
				\end{figure}
			\end{minipage}
		\end{tabular}\\
		\\
		\small
		{
			Modul: Intelligent Software Systems\\
			Topic: A survey of stochastic optimization techniques for the unit commitment problem\\
			
			Supervisor: Ogun Yurdakul\\
			WS 2019/20}
		
	}
	
	\title {Summary of Literature}
	\subtitle{Part 1}
	\author{Carolina Schorn (392137)}
	
	\date{\today\\*[60pt]}
	
	\maketitle
	
	\thispagestyle{empty}
	
	\newpage
	
	%Einstellungen zur Kopf- und Fußzeile
	\pagestyle{fancy}
	\fancyhead[L]{Carolina Schorn}	%Head = oben, foot = unten mit jeweiliger Option [L,C,R]
	\fancyhead[R]{Summary of Literatur, Part 1}
	
	
	\tableofcontents
	\newpage
	
	
	\pagenumbering{arabic}
	
	%___THEORIE_____________________________________________________
	
	\section{Notes to Papers}
	\label{sec:Notes to Papers}
	
	\subsection{Paper 14}
	\label{subsec:paper14}
	
	\begin{description}
		
	\item[Title of Paper] Data-Driven Adaptive Robust Unit Commitment
	Under Wind Power Uncertainty: A Bayesian
	Nonparametric Approach \cite{8603781}
	
	Adaptive Robust Optimierung
	für Integration von Windenergie
	Dirichlet-Prozessgemischmodell
	datengetriebener Unsicherheitssatz -> unsicherheitssatz ermittelt: steht für alle unsicherheiten zusammengefasst
	
	Ziel: robustes UC-Modell finden(mithilfe von Unsicherheitssatz und Windleistungsprognosen)

	
		\item[Mathematical Formulation]
		Two-Stage adaptive robust UC model Model 
		\begin{itemize}
			\item First Stage: Commitment Status of Generators
			\item Second Stage: Dispatch Decision of conventional Genrators and renewable wind power
			\item Goal: Minimize total operating cost
			\item Disadvantages: 
			\begin{itemize}
				\item No full use of complex uncertainty data information
				\item Does not account correlation
				\item Does not account asymmetry
				\item Does not account multimodal nature of wind power forecast errors
				\item Limited modeling flexibility
				\item $\rightarrow$ Remedy: Data driven adaptive robust unit commitment optimization framework
			\end{itemize}
		\end{itemize}
		\item[Data-Driven Adaptive Robust Unit Commitment Optimization Framework] und Text dahinter
		\begin{description}
			\item[Dirichlet Process Mixture Model]
			\begin{itemize}
				\item stochastic process
				\item Probabillity distirbution over distributions
				\item Dirichlet distributed finite dimensional marginal distributions
				\item Motivation:
				\begin{itemize}
					\item model distributions over observed data
					\item unbounded complexity: underfitting is mitigated
				\end{itemize}
				\item limited by the fact that generalizations from it are discrete distributions
			\end{itemize}
		
			\item[Data-Driven Uncertainty Set] Based on posterior predictive distribution
			\item random vector: future wind forecast errors
			\item self adaptive to underlying complexity and structure of given data
			\item[Data-Driven Robust Unit Commitment Model]
			\item[Solution Methodology]
			\begin{itemize}
				\item multilevel optimization structure \& nonconvex nature of the proposed uncertainty set $\rightarrow$ spezific solution algorythm needed
				\item reduce four-level optimazation problem into single-lebel full master problem (enumeration of all extreme points)
				\item hard to calculate (large number of induced UC contraints)
				\item $\rightarrow$ partial enumeration scheme of extreme pointss
				\item identify worst-case wind forecast error scenario
				\item compare optimal values of single subproblems to get largest one 
			\end{itemize}
		
		\end{description}
		\item[Computational Experiments]
		\item Studies on six-bus and IEEE 118-bus systems
		\begin{description}
			\item[Illustrative Six-Bus System]
			\item[IEEE 118-Bus System]
		\end{description}
	\end{description}

	
	\subsection{Paper 15}
	\label{subsec:paper15}
	
	Title of paper: A Data-Driven Model of Virtual Power Plants
	in Day-Ahead Unit Commitment \cite{8598895}
	
	\begin{description}
		\item[Motivation] Ensuring effective integration of distributed energy resources
		\item[Solution] Virtual power plants: condense them to single entity for wholesale market
		\item[Problem to solve] Dependence onn distributed pover resources output: time varying and not exactly known at day-ahead UC engine
		\item[Task of this paper] Evaluating physical characteristics of VPP 
		\begin{itemize}
			\item Max capacity
			\item Ramping capacity
			\item Encertainty in wind power output
			\item Load consumption
		\end{itemize}
	\end{description}
	
	
	
	\subsection{Paper 17}
	\label{subsec:paper17}
	\begin{description}
		\item[Motivation] Solving multistage stochastic unit commitment problem
		\item[Solution] new type of decomposition algorithm (based on new framework of staochastic dual dynamic integre programming)
		\item[label] description
	\end{description}
	
	\subsection{Paper 21}
	\label{subsec:paper21}
	\begin{description}
		\item[Problem] Uncertainty resulting from integration of variable renewable energy generations (wind-, solar power)
		\item[label] description
	\end{description}
	
	\subsection{Paper 22}
	\label{subsec:paper22}
	notes
	
	\newpage
	
	%___ZUSAMMENFASSUNg_&_ANHANG__________________________

	
	%BibTex:Bitte auskommentieren/löschen wenn es nicht benötigt wird.
	\cite{8603781}
	
	
	\bibliography{uebersicht_paper_ISS}
	\bibliographystyle {plain}
	
	
\end{document}